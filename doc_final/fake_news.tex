\subsection{Descripción del problema}
Uno de los principales efectos del reciente boom tecnológico del siglo XXI ha sido la facilidad que tiene propagar información. Mediante redes sociales, grupos de \textit{chats}, páginas web, la simplicidad con la que diferentes artículos, comentarios, imágenes, videos, etc. se difunden es asombrosa. Si bien esto es una ventaja para la globalización y la compartición de conocimiento, es también una herramienta para divulgar información no verídica que busca confundir, desorientar y, ciertamente, modificar las realidades del mundo. Quienes propagan esta información falsa no siempre son conscientes de ello, pues en la gran mayoría de los casos son personas que se fijan en la veracidad de las fuentes u otras características que evidenciarían que la información es falsa.\\

Surge entonces el problema de poder detectar estas noticias falsas, buscando detener la propagación de la desinformación que en algunos casos podría traer consecuencias realmente nefastas, en especial cuando se trata de temas de salud, como posibles curas o remedios caseros para algún tipo de enfermedad. La historia reciente de la humanidad lidiando con la pandemia del Sars-CoV-2 o COVID19 trajo, entre muchos, un caso polémico en que el Expresidente de Estados Unidos sugería la utilización de desinfectante inyectado para tratar el virus, lo cual desencadenó una serie de eventos desafortunados de personas intoxicadas por seguir la recomendación. Por otra parte, existe la reciente notificación por parte de OpenAI acerca de los resultados de generación de texto de la red GPT3 cuyos pesos e hiperparámetros no hicieron públicos justamente por la posibilidad de utilizarla para generar información falsa y posteriormente difundirla.\\

Actualmente, hay algunos \textit{datasets} diseñados para entrenar modelos para cumplir la tarea de detección de noticias falsas. Sin embargo, la gran mayoría están basados en publicaciones de redes sociales, las cuales si bien son uno de los principales medios de difusión, excluyen por completo la importante consideración a tener en cuenta respecto a la generación de noticias falsas mediante redes profundas como GPT o similares. Es por esta razón que para el desarrollo del presente proyecto se decide utilizar únicamente noticias generadas por modelos de inteligencia artificial, etiquetadas como falsas, y noticias reales recopiladas de fuentes confiables, etiquetadas como verdaderas.\\

A continuación, se describen a detalle diferentes etapas del proceso de solución al problema de detección de noticias falsas.

\subsection{Revisión del estado del arte}


\subsection{Generación de datos sintéticos}
\subsection{Extracción de características}
\subsection{Modelos de clasificación}
\subsection{Resultados}
\subsection{Conclusiones}