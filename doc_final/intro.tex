\section{Introducción}

La reciente crisis mundial ocasionada por la pandemia del COVID-19 ha dado mucho de que hablar. Según \cite{Twitter_blog}, el \textit{hashtag} \#COVID-19 ha sido mencionado por autores únicos más de 400 millones de veces. Todo tipo de temas relacionados con la pandemia: cuarentena, contagios, vacunas, crisis, economía, trabajo, salud, fallecidos, etc. han sido tendencias mundiales a lo largo del último año y medio. Este increíble volumen de flujo de información es una mina de oro para el procesamiento de lenguaje natural (NLP) y sus amplia gama de ramas de estudio, como la clasificación de texto, detección de temas tendencia, modelos de respuesta a preguntas y detección de noticias falsas.\\

Existen gran cantidad de soluciones como conjuntos (incluyendo recopilación y etiquetado de datos, métodos de extracción de características, modelos de aprendizaje automático supervisados y no supervisados, etc.) que se proponen continuamente para dar solución a los problemas que atacan las principales ramas de estudio de NLP. Es de vital importancia tener contacto con cada una de las etapas que constituyen un proyecto de \textit{Machine Learning} para poder mejorar continuamente el trabajo del área, y aprovechar un tema de tal importancia y tendencia como el COVID19 abre las puertas para ello.\\

En el presente proyecto se pretende tener un acercamiento a todas las etapas que deben seguirse para dar solución a problemas del estado del arte, como los mencionados previamente. A continuación, se detalla para cada problema (clasificación de texto, modelos de respuestas a preguntas y detección de noticias falsas) los pasos seguidos para intentar darles solución en un marco de temas específicamente limitados al COVID19 en tres idiomas diferentes con el fin de comparar la dificultad o facilidad que cada uno pueda presentar para el funcionamiento de los modelos del procesamiento de lenguaje natural. En el link \url{https://drive.google.com/drive/folders/1ZnqEWKmEq_89lxjLBC8_ZcG6qRHVQHij?usp=sharing} se encuentran todos los datos recopilados para el desarrollo del proyecto.

\newpage