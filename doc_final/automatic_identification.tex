\subsection{Descripción del problema}
Según \cite{DC_SPRINGER} la clasificación de texto se define como la asignación de etiquetas a un documento textual basado en su contenido. En este orden de ideas, este aparte del proyecto busca clasificar texto dado que se le van a asignar etiquetas a documentos de un corpus. La tarea de clasificación de texto puede ser entendida como binaria o no binaria, dependiendo de la cardinalidad del conjunto de etiquetas disponibles para la tarea. Un esquema de clasificación binario consta de dos clases únicamente, que suelen ser referidos como una clase positiva o una clase negativa. Por el contrario, un esquema no binario suele contar con más de dos etiquetas.

Dependiendo del esquema usado para la generación del modelo se pueden definir el aprendizaje supervisado y no supervisado. El aprendizaje supervisado consiste en utilizar un conjunto de datos etiquetado para entrenar el modelo. En esta esquema se suelen conocer de antemano las etiquetas a un número de documentos que se usan para el entrenamiento del modelo. En cambio, el aprendizaje no supervisado infiere las características del conjunto de datos de forma automática. En este sentido, no es necesario conocer o haber etiquetado un número de documentos de forma previa. 

A continuación se presentan dos aproximaciones para solucionar el problema de clasificación de texto bajo los esquemas de aprendizaje supervisado y no supervisado. 

\subsection{Revisión del estado del arte}
Antes de resolver el problema de clasificación de texto se procede a realizar la etapa de revisión de literatura. Dentro de la revisión de literatura se encontraron los siguientes artículos relevantes al problema de clasificación de texto:

\begin{itemize}
    \item 
\end{itemize}

\subsection{Aproximación no supervisada}


\subsection{Aproximación supervisada}


\subsection{Resultados}

\subsubsection{Aproximación no supervisada}


\subsubsection{Aproximación supervisada}


\subsection{Conclusiones}