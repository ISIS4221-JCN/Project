Dentro del análisis de los datos que se obtuvieron al aplicar el modelo planteado con los diferentes conjuntos de datos fue posible obtener las siguientes conclusiones:
\begin{enumerate}
    \item Debido a la prominencia como \textit{lingua franca} a nivel mundial, los modelos desarrollados con este idioma obtuvieron resultados más evidentes y más cercanos a lo esperado que los de los idiomas Español y Francés. Esto se puede notar con las grandes diferencias entre las palabras claves de los \textit{clusters} para el idioma inglés. Por su parte, los otros dos idiomas las diferencias no fueron tan notorias, aunque si existieron puntos que demostraron la capacidad del modelo para extraer información.
    \item Al comparar el desempeño de los modelos de español y francés se notó cómo el primero fue capaz de realizar distinciones mucho mejor que el segundo. Una situación en la cuál se puede evidenciar este fenómeno es que el modelo en español contiene menos palabras repetidas entre los \textit{clusters}. Otro aspecto a tener en cuenta es la capacidad del modelo de identificar más clusters: mientras que se identificaron cuatro en español, sólo se identificaron tres en francés.
    \item Vale la pena tener en cuenta que puede haber un sesgo hacia ciertos términos debido a la época en la cual se recolectaron los datos. Dado que estos fueron recolectados en las últimas semanas hay un sesgo a tener más términos de este periodo de tiempo a datos de periodos anteriores. Si bien se buscó mitigar este fenómeno con la normalización de los datos, es posible que aún exista un sesgo hacia estos.
\end{enumerate}
