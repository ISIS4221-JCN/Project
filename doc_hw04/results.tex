A continuación, se describen los resultados obtenidos del proceso de detección de temas tendencia con base en el procedimiento previamente descrito para cada uno de los idiomas seleccionados.

\subsection{Inglés}

\begin{figure}
     \centering
     \begin{subfigure}[b]{0.4\textwidth}
         \centering
         \includegraphics[width=\textwidth]{results/TopicDetection/en/PCA_2.png}
         \caption{2 clusters}
         \label{fig:es_kmeans_2}
     \end{subfigure}
     \hfill
     \begin{subfigure}[b]{0.4\textwidth}
         \centering
         \includegraphics[width=\textwidth]{results/TopicDetection/en/PCA_3.png}
         \caption{3 clusters}
         \label{fig:es_kmeans_3}
     \end{subfigure}
     \hfill
     \begin{subfigure}[b]{0.4\textwidth}
         \centering
         \includegraphics[width=\textwidth]{results/TopicDetection/en/PCA_4.png}
         \caption{4 clusters}
         \label{fig:es_kmeans_4}
     \end{subfigure}
     \hfill
     \begin{subfigure}[b]{0.4\textwidth}
         \centering
         \includegraphics[width=\textwidth]{results/TopicDetection/en/PCA_5.png}
         \caption{5 clusters}
         \label{fig:es_kmeans_5}
     \end{subfigure}
        \caption{Resultados de K-means con diferente número de \textit{clusters} (inglés)}
        \label{fig:en_kmeans}
\end{figure}



\begin{figure}
    \centering
    \begin{subfigure}[b]{0.49\textwidth}
        \centering
        \includegraphics[width=\textwidth]{results/TopicDetection/en/cluster0.png}
        \caption{Cluster 0}
        \label{fig:en_c0}
    \end{subfigure}
    \hfill
    \begin{subfigure}[b]{0.49\textwidth}
        \centering
        \includegraphics[width=\textwidth]{results/TopicDetection/en/cluster2.png}
        \caption{Cluster 1}
        \label{fig:en_c1}
    \end{subfigure}
    \hfill
    \begin{subfigure}[b]{0.49\textwidth}
        \centering
        \includegraphics[width=\textwidth]{results/TopicDetection/en/cluster1.png}
        \caption{Cluster 2}
        \label{fig:en_c1}
    \end{subfigure}
    \hfill
\end{figure}


\begin{figure}
    \centering
    \includegraphics[width=0.9\textwidth]{results/TopicDetection/en/cluster_over_time.png}
    \caption{Comportamiento de los clusters en el tiempo}
    \label{fig:en_time}
\end{figure}


\subsection{Español}

Para seleccionar el número de \textit{clusters} a tener en cuenta se obtuvieron las gráficas de la distribución utilizando dos, tres, cuatro y cinco grupos (\textit{véase fig \ref{fig:es_kmeans}}). La separación de los primeros tres valores (2, 3, y 4) es relativamente clara y se pueden distinguir los colores de cada uno de los grupos. Sin embargo, la distribución obtenida con cinco no permite distinguir claramente el color del quinto \textit{cluster} (\textit{véase fig \ref{fig:es_kmeans_5}}). Por otra parte, si bien es cierto que la separación para dos y tres grupos (\textit{véase fig \ref{fig:es_kmeans_2} y \ref{fig:es_kmeans_3}}) es muy clara y se ve bien distribuida, las tendencias de palabras principales y tiempo no permitían hacer un análisis claro; razón por la cual se decidió trabajar con cuatro grupos.\\

En la figura \ref{fig:es_clusters} se pueden ver las palabras con mayor frecuencia en cada uno de los \textit{clusters}. Respecto al \textit{cluster 0} se percibe cierta tendencia al inicio de la pandemia, momento en el que se empezaba a dialogar sobre la salud, contagios, trabajo y a especular sobre vacunas, conclusión respaldada por la gráfica de análisis temporal (\textit{véase fig \ref{fig:es_time}}) donde se evidencia que un alto porcentaje de documentos en los primeros meses de la pandemia pertenecen a este grupo. A medida que los meses avanzan la participación del \textit{cluster} se reduce un poco, pero sin dejar de ser importante.\\

Con relación al \textit{cluster 1}, la figura \ref{fig:es_c2} da una idea de novedades, nuevos contagios, pacientes, semana a semana, siento la salud un tema aún importante. La figura \ref{fig:es_time} muestra que este \textit{cluster} tiene una importancia relativamente baja pero constante a lo largo de todo el tiempo de pandemia, lo cual podría explicarse teniendo en cuenta que la información sobre novedades está fluyendo constantemente, sin tomar protagonismo del todo.\\

En los últimos dos \textit{clusters} se evidencia casi de inmediato una palabra que viene cobrando importancia desde hace algunos meses para acá: vacunas. Desde el comienzo de la pandemia se hablaba del tema, esperando con ansias información al respecto, pero desde que se iniciaron ensayos en humanos por parte de las farmacéuticas más importantes del mundo el tema ha sido tendencia. Se puede observar un pico en la importancia de estos \textit{clusters} hacia la mitad de 2020. Adicionalmente, en el \textit{cluster 2} se perciben aún temas como muertos, trabajo, salud, con menor importancia, que, como lo respalda la gráfica de temporalidad, han sido tendencia media pero constante durante toda la pandemia. El \textit{cluster 3} indica conversaciones sobre país, mundo, dinero, entre otros.

\begin{figure}
     \centering
     \begin{subfigure}[b]{0.4\textwidth}
         \centering
         \includegraphics[width=\textwidth]{results/TopicDetection/es/PCA_2.png}
         \caption{2 clusters}
         \label{fig:es_kmeans_2}
     \end{subfigure}
     \hfill
     \begin{subfigure}[b]{0.4\textwidth}
         \centering
         \includegraphics[width=\textwidth]{results/TopicDetection/es/PCA_3.png}
         \caption{3 clusters}
         \label{fig:es_kmeans_3}
     \end{subfigure}
     \hfill
     \begin{subfigure}[b]{0.4\textwidth}
         \centering
         \includegraphics[width=\textwidth]{results/TopicDetection/es/PCA_4.png}
         \caption{4 clusters}
         \label{fig:es_kmeans_4}
     \end{subfigure}
     \hfill
     \begin{subfigure}[b]{0.4\textwidth}
         \centering
         \includegraphics[width=\textwidth]{results/TopicDetection/es/PCA_5.png}
         \caption{5 clusters}
         \label{fig:es_kmeans_5}
     \end{subfigure}
        \caption{Resultados de K-means con diferente número de clusters}
        \label{fig:es_kmeans}
\end{figure}



\begin{figure}
    \centering
    \begin{subfigure}[b]{0.49\textwidth}
        \centering
        \includegraphics[width=\textwidth]{results/TopicDetection/es/cluster0.png}
        \caption{Cluster 0}
        \label{fig:es_c0}
    \end{subfigure}
    \hfill
    \begin{subfigure}[b]{0.49\textwidth}
        \centering
        \includegraphics[width=\textwidth]{results/TopicDetection/es/cluster1.png}
        \caption{Cluster 1}
        \label{fig:es_c1}
    \end{subfigure}
    \hfill
    \begin{subfigure}[b]{0.49\textwidth}
        \centering
        \includegraphics[width=\textwidth]{results/TopicDetection/es/cluster2.png}
        \caption{Cluster 2}
        \label{fig:es_c2}
    \end{subfigure}
    \hfill
    \begin{subfigure}[b]{0.49\textwidth}
        \centering
        \includegraphics[width=\textwidth]{results/TopicDetection/es/cluster3.png}
        \caption{Cluster 3}
        \label{fig:es_c3}
    \end{subfigure}
    \caption{Principales palabras de cada uno de los clusters}
    \label{fig:es_clusters}
\end{figure}

\begin{figure}
    \centering
    \includegraphics[width=0.9\textwidth]{results/TopicDetection/es/cluster_over_time.png}
    \caption{Comportamiento de los clusters en el tiempo}
    \label{fig:es_time}
\end{figure}

\subsection{Francés}