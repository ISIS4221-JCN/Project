La clasificación de texto es un área de investigación del procesamiento de lenguaje natural que se define como el problema de identificar y asignar una serie de categorías predefinidas a un conjunto de archivos. Según \cite{mokey_learn}, tiene un amplio campo de aplicaciones como análisis de sentimientos, etiquetado de temas, detección de \textit{spam}, detección de postura, entre otros.\\

La famosa red social \textit{Twitter} implementa algoritmos de clasificación de temas en sus \textit{tweets} con el fin e resaltar los temas más discutidos momento a momento. No obstante, una de las funciones más útiles de estos algoritmos es para etiqueta de datos. Actualmente, la inteligencia artificial y el procesamiento de lenguaje natural funcionan especialmente bien cuando se pueden entrenar los modelos sobre datos etiquetados. No obstante, la etiqueta de dichos datos es un proceso laborioso y tedioso en especial cuando se tiene una gran cantidad de datos (lo cual es ideal para mejorar el desempeño del modelo).\\

Se han propuesto numerosas técnicas y algoritmos que buscan clasificar texto, en especial enfocándose una correcta extracción de características, haciendo uso de diferentes métodos y modelos del estado del arte. A continuación, se presenta el trabajo relacionado, la metodología y procedimiento a implementar en este proyecto y finalmente resultados y conclusiones sobre el \textit{set} de datos utilizados.

\newpage