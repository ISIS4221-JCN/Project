Tratando explícitamente el tema de detección de temas en \textit{Twitter} los autores de  \cite{ReviewApproachesTopcicDetection} realizan un recuento de diferentes métodos que buscan solucionar el problema. Inicia por explicar acercamientos tradicionales que varían en la manera en que se representan los temas que son tendencia: la estrategia pivot de documentos los representa como un agrupamiento de documentos mientras que pivot de características los representa como un agrupamiento de palabras claves.


\begin{itemize}
    \item In \cite{ReviewApproachesTopcicDetection} a review of different approaches for topic detection in Twitter is presented (2020):
    \begin{itemize}
        \item \textbf{Traditional medias:} Document-pivot vs. feature-pivot.
        
        \item \textbf{Embeddings:} without (BOW, Tf-idf, NER enhanced, Metadata, LDA, etc.) vs. with (Word2Vec, pretrained word embeddings, contextual word embeddings).
        
        \item \textbf{Detection Techniques:} Classification based (supervised) and Clusstering based (unsupervised).
    \end{itemize}
    
\item \cite{TrendTopicsDetectionFromTwitter} and \cite{FuzzyIncrementalTopicDetection} present different approaches to topic clustering and trend description (extracting keywords from documents).
    
\item  \cite{DeepRepresentationClusteringTweets} and \cite{UnsupervisedDeepEmbeddingClustering} showed that encoder representations (from trained autoencoders) can improve significantly the clustering task.
\end{itemize}