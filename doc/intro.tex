El procesamiento de lenguaje natural (NLP) por sus siglas en inglés, es un campo de estudios en donde se fusionan la inteligencia artificial, las ciencias de la computación y la lingüística. Se encarga del diseño, estudio y formulación de mecanismos computacionalmente eficientes que permita a las máquinas comprender, interpretar y manipular el lenguaje humano.\\

Dentro de este campo se reúnen muchas tareas y ramas que se aplican a diferentes circunstancias en que se utiliza el lenguaje, como por ejemplo, recuperación de información, análisis de sentimientos, traducción en tiempo real, entre muchos otros. La diversidad de tareas para las que NLP puede resultar útil ocasiona que sea uno de los campos de estudio más valorados actualmente.\\

El tema que más influencia ha tenido en la humanidad a lo largo del último año es sin lugar a dudas la pandemia del COVID-19, un virus por su facilidad de contagio y grado de mortalidad ha modificado y restringido la vida cotidiana de los seres humanos en todo el mundo.
Se han declarado cuarentenas en la gran mayoría de países y si ha alcanzado un gran total de 130 millones de contagios y más de 2.83 millones de muertes \cite{WorldMeter}.\\

Otro fenómeno no tan mortal que ha modificado el comportamiento del ser humano es el acceso a Internet. Más de 4.6 billones de personas de personas tienen acceso a internet de los cuales 4.2 billones son activos en redes sociales \cite{SocialNetworks}. La intersección de estos dos fenómenos ocasiona que continuamente se genere un alto volumen de datos respecto al coronavirus, sean reacciones, noticias, comentarios, quejas, imágenes, etc. formando una generosa base de datos, especialmente para procesamiento de lenguaje natural.\\

Con esto en mente, se propone iniciar un proyecto buscando analizar y dar forma a la discusión pública en torno al COVID-19 . Para ello se inicia recopilando datos que puedan ser útiles. A continuación, se explica detalladamente las fuentes, el proceso de recopilación, el formato de almacenamiento y demás detalles que puedan resultar pertinentes para la utilización de este dataset.