\section{Formato de almacenamiento}

\subsection{Organización de los documentos}
El formato seleccionado para almacenar cada uno de los archivos es JavaScript Objetct Notation (JSON) gracias a su facilidad de almacenamiento y lectura con librerías compatibles con Python. La distribución de los archivos recuperados costa de carpetas de acuerdo al idioma: 'en', 'es' o 'fr'. Dentro de cada una de las carpetas se encuentran los archivos de cada fuente con el nombre del origen en mayúscula seguido de un guión bajo y un identificador: \textit{TWITTER\_1.json}, \textit{BBC\_1.json}, \textit{REDDIT\_1.json}.\\

A continuación, se muestra un ejemplo de un tweet.

\begin{lstlisting}[language=json,firstnumber=1]
{
    "source": "twitter",
    "url": "https://twitter.com/anyuser/status/1377678719272284164",
    "lang": "en",
    "date": "2021-04-01 17:46:17",
    "author": "SteinbachOnline",
    "text": "COVID-19: Another Death In Steinbach https://t.co/CZWur6W0Mo"
}
\end{lstlisting}

Cada archivo tendrá como mínimos la siguiente información:
\begin{itemize}
    \item \textit{source}: corresponde a la fuente de donde se obtuvo el archivo.
    \item \textit{url}: dirrección que redirigja al archivo original.
    \item \textit{lang}: idioma en que se encuentra escrito el documento.
    \item \textit{date}: fecha en que fue puclicado el documento.
    \item \textit{author}: autor del documento.
    \item \textit{text}: texto que compone el documento recuperado.
\end{itemize}

No obstante, dependiendo de la fuente algunos de estos documentos puede que contengan más información, como:
\begin{itemize}
    \item \textit{id}: identificación única del documento (especial para \textit{Reddit} y \textit{Twitter}).
    \item \textit{subreddit}: canal de donde procede la información especifica la post de reddit.
    \item etc.
\end{itemize}

\subsection{Lectura de los archivos}
A continuación, se muestra como leer fácilmente un archivo \textit{json} utilizando Pyhton.

\begin{lstlisting}[language=Python,firstnumber=1]
import json

with open('data.txt') as json_file:
    data = json.load(json_file)
\end{lstlisting}